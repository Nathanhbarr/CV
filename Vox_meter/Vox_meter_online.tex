\documentclass[10pt,a4paper]{article}
\usepackage[latin1]{inputenc}
\usepackage{amsmath}
\usepackage{amsfonts}
\usepackage{amssymb}
\usepackage{graphicx}
\begin{document}
	
Hvad er �rsagen til, at du �nsker at arbejde som analysekonsulent? 

Jeg syntes at arbejde med data er interessant. Jeg kan godt lige at dykke ned i en dataset og belyser problemstillingen og trends ved hj�lpe af statisk analyse. Jeg elsker at arbejde tv�rfagligt og skaber IT l�sninger.  

Beskriv venligst, hvem Voxmeter er, og hvad Voxmeters kunder k�ber af Voxmeter? 

Voxmeter er et analyseinstitut som leverer borger og kunder tendenser til virksomheder ved bruge af interviews, analyser og IT l�sninger. Voxmeters filosofi er igennem data analyse kan en virksomhed skaber v�rdi og v�kst. Voxmeter ligger b�de v�gt p� udarbejdning af strategier og analyser for virksomheden og tilbyder forskellige former for interviewunders�gelse og Catglobe.  

Beskriv venligst, hvorfor Voxmeter lige netop skal v�lge dig til stillingen som analysekonsulent?

Jeg har et st�rkt fundament i matematik som g�r mig i stand at bearbejde, og forst�r datasets og uddrager konklusioner. Jeg kan formidler den analyse process og konklusionen b�de skriftligt/grafisk i formen af rapportering og mundtligt.   

Skriv lidt om dig selv - hvem du er, og hvad du kan: 

Jeg er en optimistisk og rar person, som leder efter udfordringer og ansvar. Jeg har nemt at l�re og kan nemt skabe overblik over nye projekter. Jeg bruger min fritid p� at bedre kende programmerings sprog python og f�r frisk luft p� l�be ture.

X-aksen beskriver �rstider fra vinter 2005 til efter�r 2014. Y-aksen beskriver opsparing i procent. Der er tegnet to linje grafer, den gul linje graf betegner indskudt og den bl� betegner planlagt. Man kan se at den planlagt opsparing plejer at v�re st�rre end den indskudt opsparing, men ikke i perioden efter�r 2009 til vinter 2010. Tendens af de to linje grafer f�lger hinanden men den indskudt linje graf divergerer fra tendensen fra sommer 2009 til vinter 2010. Man kan ogs� se fra de grafer at der er en delay mellem de to. Den gul linje graf regerer f�r den bl�.

Tabellen beskriver den procent del af borger i forskellige regioner i Danmark der er i gang eller komme til at renovere en del af deres bolig. Man kan se at ombygninger af k�kkenet og badev�relset er prioritet h�jt. Man kan ogs� se at vedligeholdelse af vinduer, tag, gulv og loft forg�r udenfor Region Hovedstaden.

\end{document}