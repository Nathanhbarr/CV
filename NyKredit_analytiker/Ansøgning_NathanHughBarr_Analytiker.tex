\documentclass[12pt,a4paper]{letter}
\usepackage[utf8x]{inputenc}
\usepackage{ucs}
\usepackage{amsmath}
\usepackage{amsfonts}
\usepackage{amssymb}
\usepackage{graphicx}
\usepackage{hyperref}
\usepackage[margin=0.5in]{geometry}
\signature{Nathan Hugh Barr}
\address{Nathan Hugh Barr \\ Østerbrogade 202 4.tv \\ 2100 København \\ Danmark}
\begin{document}

\begin{letter}{}
\opening{\textbf{Ansøgning: Analytisk holdspiller til Realkreditprodukter i Funding \& Kapital}} 

Jeg er en nyuddannet Cand.Scient i Fysik og Matematik og ser mig selv som jeres nye analytiker, med erfaring indenfor programmering, statistik, beregningsfysik og tværfagligt samarbejde. Jeg har altid haft en stor interesse for data, og bliver motiveret af at uddrage ny viden og indsigt fra data.

Jeg søger stillingen som analytiker til Realkreditprodukter i Funding \& Kapital, da jeg finder et job med muligheden for at have ansvar for beregninger, analyser og formidlingen af data, yderst interessant. Med min passion for data og gode erfaring med problemorienteret projektarbejde, ser jeg mig selv som den oplagte kandidat til stillingen. Min faglige profil passer godt med de nævnte opgaver i opslaget, da jeg har gode kompetencer indenfor projekt orienteret arbejde og ad hoc-opgaver indenfor Matematik. Udover det, har jeg også god kompetence indenfor beregningsfysik, hvor de resultater jeg har produceret skulle være fejlfrit. Jeg kan arbejde selvstændig, hvor jeg prioriterer at skabe en sammenhæng og forståelse ud fra data. Når jeg arbejder på et projekt, er jeg struktureret, selvdrevet og jeg planlægger min tid fornuftigt, så jeg kan nå deadlines.      

I flere af disse tværfaglig projekter har jeg arbejdet med forskellige typer af datasæt indenfor fysik og biologi. I en stor del af projekterne, har jeg været i dialog med forskerne, hvor jeg har formidlet resultater og har udviklet videre på problemstillinger, der kunne undersøges. I de projekter har jeg gennemført dataanalysen ved hjælp af programmering. Eksempelvis har jeg, i min specialeafhandling, skulle programmere scripts og implementere en matematisk model, der kunne håndtere analysen af store mængder data. Jeg har erfaring med programmering i Matlab, Octave og Python, og har efter uddannelsen selv læst op på SQL, R og hvordan man kan implementere machine learning i python. 

Fra min tværfagligt uddannelse har jeg en værktøjskasse af metoder der kan bruges i projekter. Jeg kan udføre en del statistiske undersøgelse på datasæt, og med min matematisk fundament er jeg i stand til at lære nye metoder hurtigt fra teori og videnskabelige artikler. Nogle af mine styrker er, at jeg er analytisk og har en struktureret tilgang til problemstillinger. 
 
Udover uddannelsen har jeg haft forskellige studiejobs, der har givet mig god erfaring indenfor undervisning og planlægning. Jeg har gode formidlingskompetencer, og er tryg ved at stå foran mange mennesker og holde oplæg.

Her til sidst, vil jeg give et overblik over hvordan jeg er som en ansat. Mit modersmål er engelsk, så jeg kan kommunikere både på dansk og engelsk. Jeg er en ambitiøs og optimistisk person som er klar til at hjælpe når som helst. Jeg er lærenem og effektiv når jeg arbejder på opgaven, og jeg er ikke bange for udfordringer og ansvar. 

Jeg ser frem til at høre fra jer.
\closing{Venlig hilsen,}

\end{letter}


\end{document}